\documentclass[a4paper,12pt]{article}
\usepackage{amsmath,amssymb}
\usepackage{tikz}
\usepackage{fullpage}
\usepackage{bm}
\usepackage{float}
\usetikzlibrary{bayesnet}
\DeclareMathOperator{\E}{\mathbb{E}}

\begin{document}

The simplest form of the standard $3-$stage hierarchical model in the context of MEP size modeling can be described as follows. Let there be $N_P$ exchangle sequences $\{(x_i^j, y_i^j)_{i=1}^{n_j} \mid j = 1, 2 \ldots N_P\}$ of MEP size $y_i^j$ recorded at stimulus intensity $x_i^j$ from $j^{th}$ participant, for  a total of $N_P$ participants.\\

The first stage of hierarchy is the \textit{participant-level} and specifies parametric model $P(y_i^j \mid x_i^j, \theta_j)$ for each of the $N_P$ sequences and models the MEP size $y_i^j$ as a function of stimulus intensity $x_i^j$ and participant specific parameters $\theta_j$. In the second stage, $\theta_1, \theta_2 \ldots \theta_P$ are  assumed to be exchangeable and generated from  a common distribution P($\theta_j \mid \gamma$) with hyper-parameters $\gamma$. In the third stage, the hyper-parameters $\gamma$ are assumed to be unknown and are assigned a non-informative prior density $P(\gamma)$.\\

\begin{align*}
y_i^j \mid x_i^j, \theta_j, \gamma &\sim P(y_i^j \mid x_i^j, \theta_j, \gamma) \tag{4.3.1} \\
\theta_j \mid \gamma &\sim P(\theta_j \mid \gamma) \tag{4.3.2} \\
\gamma &\sim P(\gamma) \tag{4.3.3}
\end{align*}

Figure $4.3.1$ shows the graphical representation of the basic hierarchical model implemented by hbmep (Section $5.1$) for recruitment-curve fitting\\

\begin{minipage}{.5\textwidth}
    \begin{figure}[H]
        \centering
        \tikz{ %
            % Obs
            \node[obs,fill] (obs) {$y_i^j$};%
            % Deterministic Gamma mean and scale
            \node[det,above=of obs,xshift=-.8cm,fill] (beta) {$\beta_i^j$}; %
            \node[det,above=of obs,xshift=.8cm,yshift=.5cm,fill] (mu) {$\mu_i^j$}; %
            \plate[inner sep=.25cm, yshift=.2cm] {plate_obs} {(obs)(mu)(beta)} {$i = 1, 2 \ldots n_j$};
            % \plate[inner sep=.25cm, yshift=.2cm] {plate_participant_2} {(plate_obs)} {$j = 1, 2 \ldots N_P$};
            % Edges to obs
            \edge {beta,mu} {obs};
            \edge {mu} {beta};
            % % PRIORS
            % a and b
            \node[latent,above=of obs,yshift=2.8cm,xshift=-1.3cm,fill] (a) {$a_j$};
            \node[latent,right=of a,xshift=-.4cm,fill] (b) {$b_j$};
            % L, H and v
            \node[latent,right=of b,xshift=-.7cm,fill] (L) {$L_j$};
            \node[latent,right=of L,xshift=-.9cm,fill] (H) {$H_j$};
            \node[latent,right=of H,xshift=-.9cm,fill] (v) {$v_j$};
            \node[latent,right=of v,xshift=-.9cm,fill] (ell) {$\ell_j$};
            % g1 and g2
            \node[latent,left=of a,xshift=.3cm,fill] (g2) {${g_2}_j$};
            \node[latent,left=of g2,xshift=.9cm,fill] (g1) {${g_1}_j$};
            % Participants plate
            \plate[inner sep=.25cm, yshift=.2cm] {plate_participant} {(a)(b)(L)(H)(v)(ell)(g1)(g2)(plate_obs)} {$j = 1, 2 \ldots N_P$};
            % % HYPER-PRIORS
            % a_mean, a_scale, b_scale
            \node[latent,above=of a,xshift=-.4cm,fill] (a_mean) {$\mu_a$};
            \node[latent,above=of a,xshift=.4cm,fill] (a_scale) {$\sigma_a$};
            \node[latent,above=of b,fill] (b_scale) {$\sigma_b$};
            % g1_scale, g2_scale, L_scale, H_scale, v_scale, ell_scale
            \node[latent,above=of g2,fill] (g2_scale) {$\sigma_{g_2}$};
            \node[latent,above=of g1,fill] (g1_scale) {$\sigma_{g_1}$};
            \node[latent,above=of L,fill] (L_scale) {$\sigma_L$};
            \node[latent,above=of H,fill] (H_scale) {$\sigma_H$};
            \node[latent,above=of v,fill] (v_scale) {$\sigma_v$};
            \node[latent,above=of ell,fill] (ell_scale) {$\sigma_\ell$};

            % % GLOBAL PRIORS
            \node[latent,above=of g2_scale,xshift=.05cm,fill] (g2_scale_global_scale) {$\sigma_{\sigma_{g_1}}$};
            \node[latent,above=of g1_scale,xshift=-.05cm,fill] (g1_scale_global_scale) {$\sigma_{\sigma_{g_2}}$};
            \node[latent,above=of b_scale,fill] (b_scale_global_scale) {$\sigma_{\sigma_{b}}$};
            \node[latent,above=of L_scale,fill] (L_scale_global_scale) {$\sigma_{\sigma_L}$};
            \node[latent,above=of H_scale,fill] (H_scale_global_scale) {$\sigma_{\sigma_H}$};
            \node[latent,above=of v_scale,fill] (v_scale_global_scale) {$\sigma_{\sigma_v}$};
            \node[latent,above=of ell_scale,fill] (ell_scale_global_scale) {$\sigma_{\sigma_\ell}$};

            % Edges from Global-priors to Priors
            \edge {g1_scale_global_scale} {g1_scale}
            \edge {g2_scale_global_scale} {g2_scale}
            \edge {b_scale_global_scale} {b_scale}
            \edge {L_scale_global_scale} {L_scale}
            \edge {H_scale_global_scale} {H_scale}
            \edge {v_scale_global_scale} {v_scale}
            \edge {ell_scale_global_scale} {ell_scale}

            % Edges from Hyper-priors to Priors
            \edge {g1_scale} {g1}
            \edge {g2_scale} {g2}
            \edge {a_mean, a_scale} {a}
            \edge {b_scale} {b}
            \edge {L_scale} {L}
            \edge {H_scale} {H}
            \edge {v_scale} {v}
            \edge {ell_scale} {ell}

            Edges from Priors to Deterministic
            \edge {g1, g2} {beta}
            \edge {a, b, v, L, ell, H} {mu}
        }
    \end{figure}
\end{minipage}
\begin{minipage}{.6\textwidth}
    \begin{align*}
        % & \Omega = \{ a, b, v, L, \ell, H \} \\
        &y_i^j \mid x_i^j, \Omega_0^j \sim \mathcal{G}\left(\mu_i^j\cdot\beta_i^j, \beta_i^j\right)\\\\
        &\mu_i^j = \E\left(y_i^j \mid x_i^j, \Omega_0^j\right) = \mathcal{F}\left(x_i^j \mid \Omega_0^j\right)\\\\
        &\beta_i^j = {g_1}_j + \frac{\mu_i^j}{{g_2}_j}\\\\
        & a_j \sim \mathcal{TN}\left(\mu_a, \sigma_a\right) \;\; \theta \sim \mathcal{HN}\left(\sigma_{\theta}\right) \; \forall \theta \in \Omega_1\\\\
        & \theta \sim \mathcal{HN}\left(\sigma_{\theta}\right) \; \forall \theta \in \Omega_2 \\\\
        & \mu_a \sim \mathcal{TN}\left(50, 20\right)\;\; \sigma_a \sim \mathcal{HN}\left(30\right)\\\\
        &\sigma_{\sigma_L} \sim \mathcal{HN}\left(.05\right) \;\; \sigma_\theta \sim \mathcal{HN}\left(5\right) \;\forall \theta \in \Omega_2 \setminus \{b\}\\\\
        &\Omega_1 = \{b, v, L, \ell, H\} \;\; \Omega_2 = \{\sigma_b, \sigma_v, \sigma_L, \sigma_{\ell}, \sigma_H\}
        % \bm{\sigma_L} &\sim \mathcal{HN}\left(\bm{0.05}\right),\; \bm{\sigma_\ell}, \bm{\sigma_H}, \bm{g_1}, \bm{g_2} \sim \mathcal{HN}\left(\bm{5}\right)\tag{3.2.7} \\\\
    \end{align*}
\end{minipage}
\\\\\\There are few key details here. Firstly, in (3.2.2), we model the expected observed MEP size as a rectified-logistic function of stimulation intensity. Secondly, the nuisance parameter of the Gamma distribution is a linear combination of this expected value, which allows for capturing how the spread changes with respect to MEP size. This results in narrower credible intervals for the threshold estimates.

\end{document}
